\thispagestyle{empty}
\begin{center}
\Large
\textbf{\uppercase{Abstract}}
\end{center}
\medskip
\begin{center}
\large
\singlespacing
\uppercase{Economic Implications of a Changing Yield Weather Relationship }
\end{center}
\bigskip
\noindent
\textbf{Regan Arntz-Gray}
\hfill
\textbf{Advisor:\hspace{23mm}} \\
University of Guelph, 2017
\hfill
Professor Alan P. Ker
\bigskip
\\
\\
Average corn yield in Iowa and Ontario has significantly increased, potentially suggesting a corresponding increase in input demand. This implies that the yield-weather relationship may have changed over time. Yield modelling is used to test this hypothesis, and the results suggest that the yield-precipitation relationship is dynamic. To estimate the impact of neglecting the dynamic relationship, both a static and dynamic model were used to forecast yields given simulated weather from the current climate, as well as from a potential future climate. The yield forecasts were then used to estimate and compare the resulting actuarially fair premium rates. The results showed that using the static model led to biased premium rate estimates. Expected yield was calculated under both current and future climate. The mean expected yield was higher given climate change in Ontario but lower in Iowa. In both Iowa and Ontario, the variance of expected yield was higher under climate change.

\pagebreak



