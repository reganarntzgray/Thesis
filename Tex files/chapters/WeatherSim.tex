In this chapter, the methods used to simulate potential current and future weather realizations and forecast corresponding yields are explained. The implications of assuming that the yield precipitation relationship is dynamic are investigated by considering this assumption’s effect on the estimated actuarially fair crop insurance premium rates. The prospective impact of climate change on yield is considered by comparing expected yield generated based on current and future simulated weather respectively.

The following steps were performed to accomplish this:
\noindent
\\
\textbf{\underline{ Impact of Assuming a Static Yield-Precipitation Relationship:}}
\\
\textbf{ Current Simulated Weather:}

\begin{enumerate}
    \item Simulate 100 independent years of weather data for current climate by sampling from the set of historical weather data from year 2000 onwards.
    \item Forecast yields using both the static and dynamic models given simulated current weather.
    \item Estimate the county level yield distributions based on the static and dynamic simulated yield values respectively.
    \item Use the estimated yield distribution and the average county level yield over the past 10 years to estimate expected losses. Calculate the actuarially fair premium rates given these estimates of expected loss.
    \item Test whether assuming that yields are described by the static versus the dynamic model leads to statistically different estimates of actuarially fair premium rates.
    \end{enumerate}
\noindent
\textbf{ Future Simulated Weather:} 
\begin{enumerate}

    \item Simulate 100 independent years of weather data for future climate.
        \begin{enumerate}
            \item Simulate data according to the base climate period (1980-1999) in a manner analogous to the simulation of current data corresponding to the current climate period (post 2000).
            \item Modify the daily weather realizations to correspond to future climate, assuming the median changes predicted for each region and season by 21 climate models assuming the IPCC A1B scenario.
        \end{enumerate}
    \item Create yield projections using both the static and dynamic models given simulated current weather. Assume current agricultural technology (trend) and future precipitation threshold level.
    \item Follow the procedure used before given current simulated weather, to estimate the yield distributions and the corresponding actuarially fair premium rates using the projected yields.
                
\end{enumerate}
\noindent
\textbf{\underline{ Potential Effect of Climate Change on Expected Yield:}}
\begin{enumerate}
 
    \item Generate the expected yields given current and future climate using the dynamic model. 
    \item Compare the resulting expected yields in order to consider the potential effects of climate change on yield ditributions.
\end{enumerate}

   

Weather representative of two climatic periods was simulated. The simulated weather data is generated using a daily sampling mechanism on the demeaned and standardized historical weather data, and will be described in further detail below. The estimated dynamic and static one threshold models which were presented at the end of the previous section were then used to generate yield expectations given the simulated weather. These models will furthermore be referred to as the ‘dynamic yield model’ and ‘the static yield model’ respectively. The two climatic periods considered will be referred to as the current and future periods. The first (current) period for which weather is simulated is from 2000-2013 in Iowa and from 2000-2014 in Ontario, while the second (future) period is from 2080-2099 in both Iowa and Ontario.  

Yield expectations given the current and future simulated weather respectively were generated using both the  static and dynamic yield model. An error term was added to the generated yields to represent non-weather related variance. Kernel density estimation was used to estimate the underlying distributions based on the resulting simulated yield values. The actuarially fair premium rates were then calculated given these estimated yield distributions. The resulting premium rate estimates were then compared in order to consider the effect of neglecting the dynamic nature of the yield weather relationship. 

 Expected yields were generated given future weather data (and using the future precipitation threshold), as well as current data using the dynamic yield model. The expected yields are generated assuming the current level of agricultural technology. The potential effects of a particular climate change realization on yield can then be considered by comparing the yield expectations generated under the differing climates.  

\section{Simulating Current Weather Data}

Weather data representative of current climate was simulated using the daily historical county level weather data from 2000 to 2012 in Iowa and from 2000 to 2013 in Ontario. One hundred independent year long sets of simulated daily observations of county level minimum and maximum temperatures, as well as rainfall, were created. The short time span used to draw weather observations from was chosen to reflect the climatic changes of recent times.

The data simulation was accomplished according to the procedure outlined in \cite{rajagopalan1999k}, and proceeds according to the following steps.

\begin{enumerate}
            \item Deseasonalize (standardize) historical daily weather data from year 2000 on, organize into matrix $D$.
            \item Generate a random daily weather vector, $w$.
            \item Calculate the variance covariance weighted Euclidean distance between $w$ and each daily historical weather observation in $D$. 
            \item Select the 10 observations in $D$ which are closest to this vector $w$, according to this metric, and order them.
            \item Randomly choose one of the 10 observations with probability of being chosen inversely related to the order of similarity. The daily weather vector representing the conditions on the day following the selected observation is now the new daily weather vector, $w$.
            \item Begin the process again with the new $w$ and continue until one year of data is simulated
            \item Generate 100 years of simulated data in this way.
            \item Reseasonalize data. 
\end{enumerate}


This procedure generates all variables of interest together based on the previous day's simulated weather outcome. This method, therefore, recognizes the relationship between weather outcomes on successive days, and the dependence of concurrent weather variables on one another. Within any county, the temperature and precipitation variables will be correlated with each other. It is generally assumed that temperature will be affected by precipitation.  This is evidenced by weather simulation procedures, such as that used in \cite{rajagopalan1997multivariate}, which generates precipitation independently, and the other variable of interest based conditionally on precipitation. However, there is evidence which suggests that the amount of rainfall on any given rainy day depends on other weather related factors such as humidity, which is in turn impacted by temperature. Thus, it is likely that all the weather variables for a given county day combination are correlated with one another. Additionally, the weather realizations on a given day are likely to be related to those in nearby counties. Therefore, the minimum and maximum temperatures, as well as the precipitation level, will be generated for each day for all of the counties at once based on the values for the previous day. This will preserve the relationships between the weather variables, both within and between counties. 

The simulation procedure begins by deseasonalizing the current weather data. To deseasonalize, the mean of each variable for each county day combination was computed and removed from the data using only observations from year 2000 forward. Next, the data was standardized by dividing all data points by the estimated sample variance for the relevant county and weather variable. This created a data set of observed weather outcomes which were independent of time of year. Once the data was deseasonalized, a random daily realization from a multivariate standard normal distribution was generated. The random vector represents a potential daily observation of weather in all counties in the relevant area. This vector was used as the starting point for data generation.   In order to determine the weather realization on the next day, the 10 days in the deseasonalized data set which had weather realizations which were most similar to this one were selected. These ten vectors are referred to as the ten nearest neighbours. The similarity level was determined by the Euclidean distance between the two vectors, weighted by the inverse of the estimated variance covariance matrix for the deseasonalized weather variables. One of the ten nearest neighbours was then selected with probability $\frac{\frac{1}{j}}{\sum_{j=1}^{10}\frac{1}{j}}$, where $j$ is the order in terms of distance from 1 to 10 of the neighbour. Thus, the most similar neighbours were more likely to be chosen than those that were less similar. The vector representing the next day of simulated weather was then chosen to be the vector for the day which followed the neighbour which was selected. To generate the following day’s data, the same procedure was performed using the new daily weather vector and the process was continued until an entire year was generated. By this method, 100 years of daily weather data were simulated independently based on the deseasonalized county level data from year 2000 to 2013 (to 2012 in Iowa). All of the years of weather data generated represent a potential weather outcome for the year following the final year of data, given that climate remains similar to climate since 2000. 

\section{Generating Yield Distribution Estimates Based on Simulated Data}


The simulated daily data was organized according to minimum and maximum temperature, precipitation, county, and simulated year number. This data was then used to generate annual county level weather variables for use in the yield model appropriate for each location. For each year of simulated data the county level expected yields were determined by inputting the simulated weather variables into the static and dynamic yield models respectively. In order to account for variance not captured by the model, a simulated random error needed to be added. In order to simulate the non-weather related variance in yield, the sample standard deviation of the model errors over the past 10 years was computed for each region and county in Iowa and Ontario respectively. The regional level was used in Iowa, since the Iowa yield model used regional dummies, and the county level in Ontario, where county dummies were used. The errors for all counties within each region in Iowa relevant to 2013 were assumed to have variance equal to the sample variance of the errors in that region from 2003 to 2012. Similarly in Ontario, the county level error variance was assumed to be equal to the sample variance of the model errors from that county from 2004 to 2013. Errors were simulated by generating a random vector of length 100 from the normal distribution with mean 0 and variance equal to the estimated error variance for that county. These errors were then added on to the expected yield values for that county. The yields created were then representative of potential yield realizations for each county in the year following the last year of data. These resulting simulated yields from both the static and dynamic models were then used to estimate the county level yield distributions using kernel density estimation. 

Actuarially fair premia levels were then calculated based on each estimated county yield distribution and assuming a coverage rate of 90\%. The level of production that is insured was set equal to the county level average over the past 10 years of data times the coverage rate. Given that the data used is averaged at the county level, the variance will be much lower than what would be expected for a group of individual producers. This implies that the estimated actuarially fair premia rates calculated using county level yield distribution estimates are likely to be lower than they would be for an individual and given that the number of producers is not known, the farm level variance cannot be accurately approximated. However, the resulting rates will be useful as a way of comparing the effect of assuming the static model to be true as opposed to the dynamic model in terms of the ratio of the static model premia estimates to the dynamic model premia estimates. Assuming that the dynamic model is the ‘true’ yield model, the difference between the static and dynamic premia estimates will demonstrate how neglecting the dynamic yield precipitation relationship affects the ability to predict yield variance and the associated risk. 

As mentioned above, the actuarially fair premia rates were calculated assuming that the yield distribution was equal to that which was estimated using non-parametric kernel density estimation on the simulated yield values. Non-parametric techniques of estimation do not make assumptions about functional form, and are based on the sample data. Kernel density estimation allows for continuous distribution estimates based on smoothing of the sample data \citep{ker2000nonparametric}. The yield density based on the simulated data was estimated separately for each county. The basic form of the estimator is shown below.

\begin{equation}
    \hat{f}(y)=\frac{1}{nh}\sum_{i=1}^{n}K\left(\frac{y-Y_i}{h}\right)
\end{equation}

Where $\hat{f}(y)$ is the estimated probability density function (pdf) for yield, $Y$ is the vector of yield realizations where subscript $i$ refers to a particular yield in the sample data, $K$ is the kernel function, and $h$ is the bandwidth. The kernel function works to assign weight to the probability density at a given point based on its proximity to sample data points. If a yield value at which the pdf is being estimated is close to many of the yield values observed in the sample data, one would expect that the probability density at this point should be relatively high. Since the pdf estimate is generated by summing the value of the kernel function evaluated at the difference between $y$ and each of the sample points divided by $h$, the kernel function should increase for small values and decrease for large values in order to fulfill this intuitive property of a pdf estimator. Therefore, the kernel function should have the property that it increases near 0 and decreases towards both negative and positive infinity. The kernel $K$ must also fulfill the condition that it integrates to 1. Commonly, symmetric probability density functions are used as the kernel \citep{ker2000nonparametric}, and the standard normal probability density function was selected for estimation of the county level yields in this case. Generally, the choice of underlying kernel function does not have a large impact on the resulting estimate of the pdf. The bandwidth parameter $h$ however, controls the degree to which the data is smoothed and will determine the relative impact of the individual data points on the density estimate. The larger h is, the more the data will be smoothed. To select the bandwidth for this application, Silvermans' modified rule of thumb is used. Silvermans' rule of thumb yields the bandwidth which minimizes the mean integrated squared error given that the kernel used is that of a normal pdf, and when the true underlying distribution is a normal density. This rule of thumb value is given by $h=1.06\sigma n^{\frac{-1}{5}}$, where the variance of the underlying distribution  is $\sigma^2$ \citep{silverman1986density}. Given that yields can often be negatively skewed, the argument for normality of the underlying density in this case is not well supported, and using the rule of thumb parameter on non-normal data can lead to an oversmoothed estimate. However, based on empirical investigations Silverman suggests a modified rule of thumb bandwidth given by $h=0.9An^{\frac{-1}{5}}$, where $A=min(\hat{\sigma}^2,(\text{interquartile range}/1.34))$, which works relatively well in a wide range of applications and which will be used in the estimations of yield densities to follow \citep{silverman1986density}. In this case, the simulated yields from each county are assumed to be part of the same yield distribution, so the sample data, $Y$ in equation 5.1, which is used to estimate the pdf for yield in any county are the 100 simulated yields for that county only.

\section{Using Estimated Yield Distributions to Calculate Actuarially Fair Crop Insurance Premium Rates}

There are several types of crop insurance available to producers in North America which work to stabilize or guarantee levels of production, profit margins, or farm income. Given that yield is being modelled, the most applicable type of crop insurance to consider for this study is that which guarantees production level. Crop insurance providers charge a premium to producers based on factors such as: past yield realizations, level of coverage, past program claims, premiums the previous year, and the level of the reserve fund from which payouts can be drawn from. The actuarially fair premium rate is achieved when the expected profit for the insurance provider is 0. This occurs when the amount received in premiums and government subsidies is equal to the expected payouts. 

The PI and YG programs both operate such that a payout occurs if yields fall below some guaranteed level. This guaranteed level is determined by the selected coverage level and the producer's yield history. In Ontario, a producer chooses their coverage level as a percentage of their average farm yield (AFY). Similarly in Iowa, the coverage level is chosen as a percentage of the producer's actual production history (APH).
 
In Ontario, corn producers using the production insurance plan can select coverage levels of 75, 80, 85, and 90\%. They will receive a payout in the event that their actual yield is below their coverage level multiplied by their AFY, the payout threshold. The yield guarantee plan in Iowa allows the producer to select levels of coverage between 50 and 85\% in 5\% increments. Given that the premium rate calculations used in practice are based on the premium rates from the previous year, individual farm characteristics and level of the reserve fund, they were not replicated in this research as they were not within the scope of this paper. Instead, the actuarially fair premium rates were estimated based on the simulated yield values for each county. The APH and AFY were assumed to be equal to the county level average yield over the previous 10 years (from 2003-2012 in Iowa and 2004-2013 in Ontario), and the yield distributions were were estimated using kernel density estimation on the forecasted static and dynamic yields for each county given current climate. The resulting estimates would be equivalent to the actuarially fair premium rate for a producer whose APH or AFY was equal to the county level average over the past 10 years, and whose yield distribution was equal to that of the county level average yield. 

The expected profit for the insurance provider is equal to the total amount received from premiums minus the expected payouts. The expected payout is equal to the probability of a payout times the payout amount. In this case the payout amount is not fixed but depends on the overall yield level and the insured price. The payout for a given yield realization is equal to the difference between the payout threshold and the actual yield times the insured price. Therefore, the expected loss is equal to the integral of the probability density function for yield times the difference between yield and the payout threshold times the insured price, evaluated from 0 to the payout threshold. The total collected in premiums is equal to the premium rate times the insured amount times the insured price. Therefore, the expected profit is as below:

\begin{equation}
    E(\pi)=rip-\int_{0}^{i}(f(y)(i-y)p)dy
\end{equation}

Where $\pi$ is profit, $i$ is the insured amount or the payout threshold, $y$ is a realized yield, $f(y)$ is the estimated yield probability density function for the relevant county, $r$ is the premium rate and $p$ is the insured price. For this study it is assumed that the coverage level is 90\%, so $i$ is equal to the AFY or APH times 0.9 in Ontario and Iowa respectively. Setting equation 5.2 equal to 0 and solving for $r$ gives equation 5.4 below.

\begin{equation}
    rip=\int_{0}^{i}(f(y)(i-y)p)dy=p\int_{0}^{i}f(y)(i-y)dy
 \end{equation}   
 
  \begin{equation}
    r=\frac{1}{i}\int_{0}^{i}f(y)(i-y)dy
\end{equation}

Using $i$ equal to 0.9 times the average county level yield over the past 10 years, and estimating the pdf for the yields generated with the dynamic and static models respectively, the actuarially fair premium rates for each county, under each model can be estimated. The resultant premium rates given the static versus the dynamic model can then be compared. This will inform the degree to which the changing yield weather relationship over time affects the ability to estimate expected losses.

Given that the estimates of the pdfs for county level yields are not based on a known parametric form, they could not be integrated exactly in order to solve the above equation. Instead, these integrals were estimated by Riemann sums. For each county the kernel density estimate of the county level yield pdf was evaluated at 1000 equally spaced points, $y_n$, between 0 and $i$ and multiplied by the difference between i and $y_n$. These 1000 results were summed and multiplied by the width of the space between each point in order to estimate $\int_{0}^{i}f(y)(i-y)dy$.


 

\section{Simulating Future Weather Data}


Projections for changes in regional temperature and rainfall patterns, generated based on Intergovernmental Panel on Climate Change (IPCC) scenarios, were used to create simulated future weather data. The IPCC has several future scenarios and models which result in different climate trajectories. The projections used were based on the A1 scenario which considers the future climate given ``A future world of very rapid economic growth, global population that peaks mid-century and declines thereafter, and rapid introduction of new and more efficient technologies." The A1 scenario has three different branches - one in which energy is mainly derived from fossil fuels, one in which mainly non-fossil fuels are used, and one in which there is a balance across both sources \citep{IPCCscenarios}. This balanced scenario is referred to as A1B, and is the scenario that will be used for creating simulated future data. Given a scenario or trajectory for the future, the expected magnitude and direction of change will depend on the model used to create the projection. A set of 21 global climate models archived at the Program for Climate Model Diagnosis and Intercomparison were used to predict regional changes in temperature and precipitation during each quarter of the year by 2080-2099 relative to 1980-1999. The results from these 21 models were used to determine the median, 1st quartile, and 3rd quartile expected changes for temperature and precipitation for each region and time of year, based on the results from the 21 models. For the future weather data simulations used in this study, the projections corresponding to the median percentile changes projected by the 21 models will be used. To simulate future weather, data for the base period of comparison is first generated. In this case the base period is 1980-1999, since the projected changes in temperature and precipitation by the 2080-2099 period are relative to this time. In an analogous way to that in which current weather data was simulated using the daily county level data from year 2000 forwards, data relevant to the 1980-1999 climate was simulated using the historical daily weather data from this period. 

To create the future simulated data, the median predicted effects of climate change, given the A1B scenario, were used to modify the simulated base period weather. The projected changes in temperature by region and season were added to the maximum and minimum daily temperature values, and the projected percentage change in precipitation was multiplied by the simulated daily precipitation values. For example, the median projected temperature change in the Eastern North American region, which encompasses Ontario, during the months of June, July and August is 3.3 degrees Celsius. Using the simulated base data, 3.3 is added to the values which fall in these months in order to adjust the simulated weather to match the future projected changes. Similarly, the median projected increase in precipitation by the 2080-2099 period is 1 percent during June, July, August in this region. Thus, the precipitation values will be multiplied by 1.01 in order to adjust them to the increased levels of rainfall which are projected for that time period and region.

\section{Generating Yield Distributions and Estimating Actuarially Fair Premiums Rates Given Future Climate}

The static and dynamic yield models were used to generate expected yields given the future simulated weather. The precipitation thresholds were set at the 2090 level, given that the future weather data was relevant to the period from 2080-2099. Given that the rate of future technological advancement over the next 75 years in agriculture is not known, the trend variable is set at the level corresponding to 2013 in Iowa, and 2014 in Ontario. The amount of non-weather related variance in yield in the future is also unreasonable to forecast at this large of a distance in time, and so it is assumed to be equal to the current error variance. Therefore, the error variance estimates used for current yield generation are used for the future yield generation as well. Given the yields generated under the static and dynamic models given future climate, the actuarially fair levels of crop insurance premiums are estimated. The estimated premium rates resulting from the use of the static model to project yields, can then be compared to those estimated assuming the dynamic model. 

To consider the potential effects of climate change on the mean and variance of yields, the dynamic model is used to compare expected yields under current and future climate. The expected yield based on the dynamic model was calculated with the current simulated weather and current threshold level, as well as with the future simulated weather and future threshold level. No additional variance term was added given that the effect on expected yields was of interest. Therefore, the expected yields are calculated using the dynamic yield model with both current and future simulated weather, given the current level of technological advancement, and without adding in an additional error term. The variance of the expected yields presented is very likely smaller than the true variance would be as only variance from weather as measured by the dynamic yield model is considered. However, the difference in the variance of the expected yields under current versus future simulated weather represents the changes in mean yield and yield variability which could be expected due only to the changes in climate, all else being equal.
