
\section{Background and Motivation}

Agricultural production depends on several factors, of which only some are within a producer's control. While farm management choices are important to the success of crops, weather is a significant and uncontrollable determinant of crop yields. Farm incomes depend on yield which is inherently uncertain due to its reliance on weather. Farm incomes also depend on prices at the time of harvest, which further contributes to income uncertainty. The level of uncertainty faced by producers creates a market for income risk-reducing mechanisms such as crop insurance. Crop insurance programs are heavily subsidized in both the United States and Canada \citep{rosenzweig2002increased}. This effectively transfers a large portion of producer income risk to the public. In 2014 the total amount spent on risk management programs such as crop insurance by the public was nearly \$1.4 billion in Canada, while in the United States \$6.2 billion was spent on premium subsidies \citep{kerRMP2016}. Subsidized crop insurance programs comply with trade deals, and are therefore a popular method of transferring income to agricultural producers. These programs are subsidized to encourage producer participation in order to avoid the need for short-term disaster assistance programs \citep{coble2013we}. Historically, significant premium subsidies have had to be in place for participation to significantly increase \citep{goodwin2013harm}. 

Agricultural technology over the past 60 years has focused on increasing mean yields, as opposed to decreasing production risk. Corn yields in Iowa and Ontario increased more than three-fold over that time period \citep{CANy,NASSy}. As mentioned above, producers generally do not bear all of their income risk, due to the subsidization of crop insurance programs. This alters the rational production decision set, as farmers are less sensitive to the negative impacts of highly variable crop varieties. This could lead to producer demand for varieties with high mean yields, even if these high yields are attained at the expense of increased variability. Additionally, the change in the yield distribution in Ontario was found to be empirically consistent with the use of subsidized crop insurance as a substitute for other income risk reducing mechanisms \citep{kerRMP2016}. 

The large increase in mean yield since the 1950's was driven by several factors: increased attention to farm management practices such as time of planting, a large increase in nitrogen application from the 1950's to the 1980's, and a steady increase in plants per acre until present \citep{duvick2005contribution}. This increase in yield, and in particular in plants per acre, suggests an increase in demand for inputs per acre. Given the dependence on weather, not all agricultural inputs are in the control of producers. In particular, water is largely out of producer control as almost all corn acres are rain fed in both Iowa and Ontario. The large changes in plants and yield per acre suggest that the relationship between precipitation and yield may have changed over time. 

In addition to these changes, agricultural systems also face the impending effects of climate change. Climate change is expected to lead to more extreme heat events as well as more extreme precipitation events, both of which have the potential to be detrimental to corn yields \citep{rosenzweig2002increased,southworth2000consequences}. Many factors can impact the way in which crop yields will respond to climate change, such as the current temperature and precipitation distribution, and the type and quality of the soil. Top soil depth has been found to correlate with soil water capacity. This implies that regions with deeper top soils will likely be less sensitive to to drought conditions, as the precipitation received can be used more effectively. Deep top soils could therefore reduce the negative impact of extreme precipitation events \cite{lee2015topsoil}. Finally, increasing world population will likely lead to a corresponding rise in global demand for field crops. This increase in demand, coupled with climate change effects could threaten global food security \citep{edgerton2009increasing}. Given these considerations, it is important to consider the fragility and productivity of our food systems \citep{williams2016soil}.


\section{Economic Problem}

 The number of plants per acre, as well as average yields, have increased throughout the study period \citep{duvick2005contribution}. The increase in yield and plant populations per acre implies a potential increase in the level of inputs per acre needed to produce a relatively high yield. Given that most corn fields are rain fed in North America, the water input is essentially fully dependent on weather conditions. If the amount of water required per acre has increased, it is possible that the relationship between precipitation and yield may have changed over time. The magnitude of this change will likely depend to some degree on conditions affecting water uptake levels such as soil quality and depth. A change in the relationship between yield and precipitation could have implications regarding mean yields and the degree of yield variability. Yield variability is inherent in agriculture due to the reliance on weather which is, by nature, unpredictable. As producers rely on yields for their income, yield variability leads to income variability and therefore risk for producers. 

In addition to the possibility that the yield-weather relationship is changing, the distributions of key weather variables affecting corn yields are likely to be modified as a result of climate change. Since crop yields are dependent on weather, changes in weather patterns will also influence yield and income variability. The negative impacts of income variation, to which producers are subject, can be partially mitigated by participation in crop insurance programs \citep{rosenzweig2002increased}. Producers are faced with decisions regarding their crop insurance plans including what aspects of the farm to insure, and the level of coverage they would like. Under some plans the producer can also choose a fixed or floating claim price; a fixed claim price is determined based on price expectations at the start of the season, while a floating claim price is determined based on actual market conditions after harvest \citep{Agri2014, NCIS}. These decisions must be made based partially on the producer's expectation of yields and yield variability. The economic problem which motivates this research is how the changes in agricultural technology, coupled with potential climate change effects, will impact corn production, producer livelihoods, and the efficacy of crop insurance programs, through their effect on the yield weather dynamic.

The impact of weather, in particular precipitation level, on corn yields and how these impacts have changed over time will be investigated. The potential yield response to predicted climate impacts will then be considered based on the results of this investigation. The solution to this problem is important to producers and crop insurance providers as well as tax payers, as it will allow for more accurate short term and long term predictions of yields, given accurate weather and climate forecasts. The improved ability to predict yields may help producers better cope with the levels of weather and climate related risk associated with crop farming. Information regarding yield forecasts under climate change could aid producers in making decisions which allow them to adapt their production systems in their best interest. Since crop insurance premia are set, in part, based on estimates of crop yield distributions, improved yield forecasts may lead to more accurate estimates of expected losses. Crop insurance providers need to maintain financial soundness over time, meaning that the total indemnities paid must equal the total value of premiums and subsidies collected. Accurate estimates of  yield distributions could aid in the ability of insurers to set premiums in such a way, and to insure that the reserve fund balance is kept at an appropriate level. Since crop insurance programs are subsidized to some degree by government, their efficient operation is desirable for the general public as well. Finally, a basic understanding of how food systems are affected by weather could assist in large scale and long term planning. This understanding could inform the best way to adapt these systems to the changing underlying conditions which may arise due to the intersection of climate change, and changes in agricultural technology.

\section{Economic Research Problem}

The economic research problem is how to construct a weather dependent corn yield model which changes through time. This model can then be used, along with simulated weather data from possible future climate scenarios, to estimate how yield may be affected under climate change, and how this will impact crop insurance rates. Solving this problem will allow for the consideration of possible consequences of the intersection of a changing yield-weather dynamic and climate change. The results of this study could be used to inform adaptation strategies at the individual as well as national levels. 

The yield-weather response function will be determined using historical data. The precipitation-yield relationship will be modelled such that the effect of precipitation on yield is dynamic through time. This is done in order to allow for, and subsequently test, the possibility that changes in agricultural technology may have modified the yield weather model. This research will aid in the understanding of the impact of weather on corn yields. This understanding, in conjunction with estimates of current and future weather distributions, could be used to inform the levels of crop insurance premia. Since crop insurance in Ontario and Iowa is subsidized by government to a large extent this research is also relevant to tax payers. The ability to forecast yield under predicted climate change scenarios will allow for the investigation of the effect of climate change on the mean and variance of expected corn yields.  Estimating the nature of climate change effects on crop yields will allow for planning and development of appropriate adaptation strategies. 

This research falls into the class of problems which deal with producer risk and uncertainty in the insurance market. The potential changes in the yield climate relationship could lead to changes in the estimated level of production risk, while the dependence on weather implies continued uncertainty. Although the proposed research focuses only on corn in Ontario and Iowa, this research question could be asked of other crops grown in different locations, and approached using similar methods. While there will likely be differences in the effects of weather on various types of crops grown in a range of climatic conditions, the overall framework of determining relationships between various weather variables and yield over time and using those results to improve information in the crop insurance market could be applied elsewhere. 

\section{Purpose and Objectives}

The purpose of this study is to determine whether or not there is evidence to suggest that the relationship between corn yield and precipitation has changed over time. The potential consequences of these results on the ability to accurately predict yield distributions will be considered in terms of its effect on the estimation of the actuarially fair level of crop insurance premia. Finally, the consequences of climate change given the findings regarding the yield weather relationship will be investigated.

The objectives of this research are:

\begin{enumerate}
 \item To review the literature in order to determine the most appropriate explanatory variables to include in a crop yield model.
  \item To develop a conceptual framework of corn yield in relation to weather variables which can change through time, and use this framework to develop an empirical model.
  \item To estimate this empirical model, and determine whether there is evidence for a changing yield precipitation relationship over time.
  \item To consider the effect that the possibly dynamic yield precipitation relationship could have on the estimation of yield distributions and the actuarially fair premia level.
  \item To consider the possible implications of these results under potential climatic changes.

\end{enumerate}
 
The yield-weather relationship will be modelled using historical data, such that the precipitation effect can change through time. The effect of particular weather variables on yield, such as total precipitation at key growth stages, will be estimated. Given simulated weather realizations from both the current climate and a potential future climate, this model will be used to forecast yields so that the possible effects of climate change on corn yield can be considered. The effect of allowing for the potentially dynamic yield precipitation relationship on the resultant yield distribution estimates will be discussed in terms of its effect on the calculation of the actuarially fair level of crop insurance premia. 

